\section{Event-B Models}
-- Notes on Event-B machine --

\subsection{HD2}
Ignore all values, focus only on the very abstract level. Don't have physical entities, like pumps that can be on or off.
HDM1 - Patient blood never goes below 1.


Fails at refinement 2. Can not clean the blood before giving it is pumped back to the patient.

HD3

HDM10
If bloodpump in on, the system increases its amount of blood currently inside the system.
Blood inside the system can be cleaned if the ultrafiltration pump is turned on and there is blood in the system.
Cleaned blood is removed from the system. IN real life it is returned to the patient. 

HDM100
For any reason, we may turn off the Blood pump

HDM1000
Can only turn on blood pump when the alarm is off.
Alarm can be turned on for any reason

HDM10000

Need a function that takes a SystemState and return its substates. Cleaning, connecting etc.
Added SystemPhaseCtx for the 3 system phases and new csystemPhase variable. Also added an event to change a the variable to different phases.
Added events that contains monitored variables that cause the blood pump to stop and activate an alarm. For now, the alarm is turned on and the BP can only be turned off when the alarm is on.


SystemPhaseCtx1
Added the SUBPHASES of the HD treatment. $$ SUBPHASE \in SystemPhases \longrightarrow SubPhases $$


SystemPhaseCtx1 + HD100000
Added SystemSubPhases to the machine that allows for changing of the software state, only forwards.

Subphases are given an order and they must follow the order when the system changes states.


HD10000000
BloodPump and requirements regarding the System state has been added. 


HD10000001
Added SAD events. Air volume limit, flow through the SAD. 


Refactored all the machines to remove UFPump and blood pump variables. BP in now abstracted. 

HD10000002
UFPump causes DF to flow and create pressure that removes toxins and water from the blood. Waste bucket contents - Used DF = water extracted.

Main problem: I do not know how the dialysis machine truly works. Why they use certain components.


HD10000003

Added bypass chamber. Left and right side of a chamber.


HD10000004
Adding Dialyzer.

Dialyzer has inner tube with blood, and the other tube with dialyzing fluid. The blood tube is encased inside the dialyzing fluid. 

Do I need to monitor that blood is in the tubes? The real system will not know. Only if the pump is on and that the VRD is detecting blood by the exit.

Note: Gradual implementation or Refined implementation: Can implement "isBloodPumpOn" and use as guard for new events(Gradual implementation). Or do it all in one single refinement that adds a guard to all events. End result is the same, but doing everything in one refinement seems like the correct approach. 

To build a machine:
Create a machine that has the required events and variables to perform a valid/safe dialysis treatment. This machine should be thought of as a state graph were events trigger transitions. Our exit nodes should contain at least one valid/safe and successful run of a treatment. Further refinements should be dedicated to trimming the graph from nodes that are illegal w.r.t requirements.

The KEEP method is very hard to do. If there exists independent components, it might be easier. In this system, all the components are linked together.  The state of one component has an effect on all other components (mostly). One could select the system that does not have any outbound dependencies, but this will require a full understanding of the system from the start. An incremental approach could allow for learning a system piece by piece. 










